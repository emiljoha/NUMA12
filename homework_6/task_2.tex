\begin{problem}
Construct a program to perform Gaussian integration, with the following input:
\begin{itemize}
\item Interval of integration
\item Function to be integrated
\item Number of terms in the quadrature formula
\end{itemize}
According to Ramanujan the number of numbers between $a$ and $b$ that
are either squares or sums of two squares is given approximately by
\begin{equation*}
0.764\int_a^b \frac{\text{dx}}{\sqrt{\ln x}}
\end{equation*}
Use your program to test this statement for $a=1$ and $b=30$
\end{problem}

\begin{solution}
  First of all, let us outline the basic theory behind the algorithm
  that we have implemented. The general formula for the Gauss
  quadrature is:
\begin{equation*}
\int_{-1}^1 f(x) \, \text{dx} \approx \sum_{i=1}^n f(x_i)w_i
\end{equation*}
We can change the interval to whatever we like by:
\begin{equation}
  \int_a^b f(x) \, \text{dx} \approx \frac{b-a}{2} \sum_{i=1}^n w_i f\left(\frac{b-a}{2}x_i+\frac{a+b}{2}\right)
\label{gaussquad}
\end{equation}
To calculate the nodes $x_i$ we take them as the roots of the Legendre
polynomial of degree $n$, which is defined as:
\begin{equation*}
P_n(x) = \frac{1}{2^n} \sum_{k=0}^n \binom{n}{k} (x-1)^{n-k}(x+1)^k
\end{equation*}
Then, we construct the Lagrange polynomials $l_i$ and compute the
weighs as:
\begin{equation*}
w_i = \int_{-1}^1 l_i \quad \forall i = 0, \ldots, n
\end{equation*}
Finally we put all of this together in equation \ref{gaussquad} to get
the approximation of the integral.

The algorithm seems simple at first but the actual implementation
posed a few challenges. We first tried to find the roots of the
Legendre polynomials exactly, but this turned to be disappointingly
slow for degree greater than 5, we overcame this by turning to
\texttt{numpy.roots} to find the roots numerically. We also did not
succeed in finding a closed form for the integrals of the Lagrange
polynomials, thus we integrated them symbolically as needed using the
python library SymPy. After these workarounds (and some other minor
ones) we finally got a working script and calculated the integral,
which turned out to be, with 15 terms in the quadrature formula:
\begin{equation*}
0.764 \int_1^{30} \frac{\text{dx}}{\sqrt{\ln x}} \approx 15.1713138457160
\end{equation*}
There exist $15$ numbers in $1, 2, \dots, 30$ which are either squares
or sums of two squares. This suggests that our method is correct and
that Ramanujan did a good job with his formula. Finally, here is our
code:
\lstinputlisting{code/task_2.py}
\end{solution}
\newpage


%%% Local Variables:
%%% mode: latex
%%% TeX-master: "report"
%%% End:
