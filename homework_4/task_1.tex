\begin{problem}
  Show that $T_n$ and $T_{n−1}$ have no common zeros.
\end{problem}

\begin{solution}
  First we assume there is an number $\theta$ such that
  equation~\ref{eq:CommonZero} holds. If there is such a $\theta$ then
  there is such a $x = \cos\theta$ so that $T_n$ and $T_{n−1}$ have
  common zeros. We will thus need to show that this leads to a
  contradiction.
  
  \begin{align}
    \label{eq:CommonZero}
    \begin{cases}
      T_n(\cos\theta) = \cos n\theta = 0 \\
      T_{n+1}(\cos\theta) = \cos (n+1)\theta = 0
    \end{cases}
    \Leftrightarrow
    \begin{cases}
      n\theta = \frac{\pi}{2} + 2\pi k \\
      (n+1)\theta = \frac{\pi}{2} + 2\pi k^{\prime}
    \end{cases} \\
    \nonumber
    \Leftrightarrow
    2 \pi k = 2 \pi k^{\prime} - \theta 
    \Leftrightarrow
    \theta = 2 \pi (k^{\prime} - k) \\
    \label{eq:contradiction}
    \Rightarrow
    \cos n\theta = {(-1)}^{n(k^{\prime} - k)} \neq 0 
  \end{align}

  The result of the assumption of existence leads to the contradiction
  in equation~\ref{eq:contradiction}. Thus our assumption must be
  false there is no $\theta$ such that it is a root to both $T_n$ and
  $T_{n−1}$. Note that not all $x$ can be written as
  $x = \cos \theta$ but only $x \in [-1, 1]$. However this is the
  interval in which Chebyshev polynomials are studied and what happens
  outside is of little importance.
\end{solution}

%%% Local Variables:
%%% mode: latex
%%% TeX-master: "report"
%%% End:
