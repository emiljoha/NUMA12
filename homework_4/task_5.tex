\begin{problem}
  In the exchange algorithm it is necessary to find the point $\eta$
  that satisfies the equation

  \begin{equation}
    |f (\eta) − p (\eta)| = \|f − p\|_{\infty}
  \end{equation}

  but in practice it is inefficient to try to calculate extrema of
  functions exactly.  Investigate useful ways to approximate $\eta$ and
  discuss your results.
\end{problem}


\begin{solution}
  There are two different approaches to finding this maxima. Firstly,
  one can use one of all the fancy optimization methods in
  existence. The other is two do a quick and dirty
  solution. Sacrificing precision for speed and simplicity.

  The most simple quick-and-dirty method is to calculate the error
  function on a grid, Choosing $\eta$ as the point on the grid that
  corresponds to the highest value of the points tested. The grid
  spacing is be chosen to ``resolve'' the changes in the
  function. What is important is to find a value close to the actual
  maxima, not to accurately find it.

  The more sophisticated approach is to use one of the many
  optimization algorithms in existence. These methods have in common
  that they are able to locate the maxima at a much higher
  accuracy. However, they also often get stuck at a local maxima which
  is of much larger concern than inaccuracy in position. However this
  do not exclude the existence of less naive brute force methods.

  The brute-force grid method sacrifice inaccuracy for simplicity and
  avoidance of getting stuck in local maxima. The maxima is not a
  significant result in itself but simply just a guess, not an answer,
  to the choice of reference. As the choice is of the maxima as the
  next point is an approximation, the precision of the actual maxima
  becomes less important. 
\end{solution}

%%% Local Variables:
%%% mode: latex
%%% TeX-master: "report"
%%% End:
