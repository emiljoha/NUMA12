%%%%%%%%%%%%%%%%%%%%%%%%%%%%%%%%%%%%%%%%%%%%%%%%%%%%%%%%%%%%%%%%%%%%%%%%%%%%%%%%%%%%
%%%%%%%%%%%%%%%%%%%free online editor available at%%%%%%%%%%%%%%%%%%%%%%
%%%%%https://www.writelatex.com/%%%%%%%%%%%%%%%%%%%%%%%%%%%%%%%%%%%%%%%%%%

\documentclass[10pt,leqno ]{article}
    %The document class defines the master templates, the structure of the document, 
    %and lays out the types of 
    %objects that can be manipulated for this type of document. 
    %the brackets contain basic options that will be applied globally (throughout
    %the document). Here, we specify a 10pt font, and when we number an equation, the 
    %number will be on the left.
    %The document class is a file ***.cls. You will probably never have to edit or create
    % a .cls file. There are many available on the internet for your use.

%%%%%%%%%%%%%%%%%%%%%%%%%%%%%%%%%%%%%%%%%%%%%%%%%%%%%%%%%%%%%%%%%%%%%%%%%%%%%%%%%%%%%%%%%%%%%%%%%%%%%%%%%%%%%%%%%%%%%%%%%%%%%%%%%%%%%%%%%%%%%%%%%%%%%%%%%%%%%%%%%%%%%%%%%%%%%%%%%%%%%%%%%%%%%%%%%%%%%%%%%%%%%%%%%%%%%%%%%%%%%%%%%%%%%%%%%%%%%%%%%%%%%%%%%%%%
\usepackage{amsfonts}
\usepackage{amssymb}
\usepackage{amsmath}
\usepackage{times}
\usepackage{amsthm}
\usepackage{hyperref}
\usepackage{homework}
\usepackage{dsfont}
    %packages control the ``style'' or look of the document. These come in the form of 
    %files ***.sty. The package ``homework'' above was created by me. The other packages
    %are very common for this type of document. You can google to learn more about what
    %they can do, and what options they give you. For example

\usepackage{graphicx}
%\usepackage{svg}
\usepackage{mathtools}
\usepackage[margin=1.5in]{geometry}
    %the geometry package lets you customize the margins of your document.
    % and the 
  
\usepackage{setspace}
    %package gives us the ability to set the line spacing.
   

\newtheorem{theorem}{Theorem}
\theoremstyle{definition} 
\newtheorem{problem}[theorem]{Task}
%these set up environments for listing things. The numbering is automatic.

    
\newenvironment{solution}[1][Solution]{\begin{doublespace}\textbf{#1.}\quad }{\ \rule{0.5em}{0.5em}\end{doublespace}}
    %this is the environment for writing solutions. Doble spaced, with an end of proof
    %box at the end
    
\title{Numerical Approximation\\
NUMA12, Spring, 2017\\
Homework 3}
\author{Anton Makarov and Emil Johansson \\
        Lund University}
    %above is the information that goes in the title. Notice the { and }. 
    %the double slashes \\ mean start a new line.


\begin{document} %this means end the preamble (stuff controling the styles above and
%start the content of the document. We can make adjustments as we go. For example,
\maketitle %make the title according to the styles outlined in homework.sty
\vskip .25in %skip a bit before we start the regular text.
\thispagestyle{empty} %no need to number first page.

\begin{problem}
  Let $-1 \leq x_0 < x_1 < x_2 \leq 1$ and $X$ denote the Lagrange
  interpolator at these points. Show that the minimum value of
  $\lVert X \rVert_\infty$ is 5/4 and is attained when $-x_0 = x_2 \geq 2
  \sqrt{2}/3$ and $x_1 = 0$. Show that if $x_0 , x_1 , x_2$ are
  the zeros of $T_3$, then $\lVert X \rVert_\infty = 5/3$.
\end{problem}

% --------------------------------------------------------------------%

\begin{solution}  
  As we have 3 interpolation points, we know that our polynomial is of
  degree at most 2. The Lebesgue constant of the interpolation
  operator can be written as:
  \begin{equation}
    \lVert X \rVert_{\infty} = \max_{x\in [-1,1]} \sum_{i=0}^2 \lvert l_i(x) \rvert
    \label{defX}
  \end{equation}
  Where $l_i$ are the Lagrange basis polynomials. The task of
  minimizing the max norm of a quantity is the task of balancing
  local maxima so that they are equal. By numerically studying the
  local maxima´s of equation~\ref{defX} we see that the as the distance between two
  points increase, so does the value of the local maxima in
  between. From this realization we see that $x_1= 0$ and
  $-x_0=x_2$ as this is the only setup that preserves the symmetry of
  equation~\ref{defX}. Any other choice would result in one side
  having a higher maxima that the other.

  Now there is only one degree of freedom left. From
  figure~\ref{fig:4edge} in task 4 we see that there is one more
  possible maxima that we have not yet considered. The end
  points. Figure~\ref{fig:4edge} suggests that the end points decrease
  as the end points go further out and become smaller than the
  internal maxima´s around 0.95. A more careful numerical calculation
  gives the results in table~\ref{tab:1}. In table~\ref{tab:1} we see
  something quite spectacular. Between $0.941$ and $0.943$ the maximum
  stop changing and remain stable to an error of $10^{-6}$ for points
  closer to the edges. This value for when the maxima become stable
  perfectly corresponds to $2\frac{\sqrt{2}}{3}$.

  \begin{table}[!ht]
    \caption{Table of global maxima of $\sum_{k=0}^n|l_k(x)|$ with
      three points. $\{-x, 0, x\}$ for $x \in [0.92, 0.96]$ with
      steps length $0.002$}
    \label{tab:1}
    \begin{center}
      \begin{tabular}{ l  c  r }
        $x$ & Maxima \\ \hline
        0.923232323232 & 1.34643211124 \\
        0.925252525253 & 1.33619686886 \\
        0.927272727273 & 1.3260284506 \\
        0.929292929293 & 1.31592627599 \\
        0.931313131313 & 1.30588977089 \\
        0.933333333333 & 1.29591836735 \\
        0.935353535354 & 1.28601150353 \\
        0.937373737374 & 1.27616862366 \\
        0.939393939394 & 1.26638917794 \\
        0.941414141414 & 1.25667262245 \\
        0.943434343434 & 1.24999993218 \\
        0.945454545455 & 1.24999937708 \\
        0.947474747475 & 1.24999992242 \\
        0.949494949495 & 1.24999941213 \\
        0.951515151515 & 1.2499999121 \\
        0.953535353535 & 1.24999944589 \\
        0.955555555556 & 1.24999990124 \\
        0.957575757576 & 1.24999947837 \\
        0.959595959596 & 1.24999988984 \\
        0.961616161616 & 1.24999950961 \\
        0.963636363636 & 1.24999987792 \\
        0.965656565657 & 1.24999953963 \\
      \end{tabular}
    \end{center}
  \end{table}

  If, as stated, we take the points $x_0 = -1, x_1 = 0, x_2 = 1$ and
  expand the LHS of equation \ref{defX} we obtain:
  \begin{equation}
    \max_{x \in [-1,1]} \left \{ \left | \frac{x(x-1)}{2} \right | + \left | -x^2+1 \right | + \left | \frac{x(x+1)}{2} \right | \right \}
    \label{eq1t1}
  \end{equation}
  We can split the inside of equation \ref{eq1t1} as:
  \[ \begin{cases} 
      -x^2-x+1  &-1< x< 0 \\
      -x^2+x+1  &0< x<1 
    \end{cases}
  \]
  Taking the derivative and making it equal to zero yields
  $x = \pm 1/2$. We also consider the problematic points $x = \pm 1$
  and $x = 0$. By substituting these values at our function we get
  that the maximum is attained at $x=\pm 1/2$ with the value 5/4,
  therefore the lower bound for the operator is attained.

  Finally we have that $T_3 = 4x^3-3x$, the roots of $T_3$ are
  $x=0, x=\pm \sqrt{3}/2$. By choosing
  $x_0 = -\frac{\sqrt{3}}{2}, x_1 = 0, x_2 = \frac{\sqrt{3}}{2}$ and
  expanding the LHS of eqaution \ref{defX} we obtain:
  \begin{equation}
    \max_{x \in [-1,1]} \left \{ \left | \frac{2x^2-\sqrt{3}x}{3} \right | + \left | \frac{-4x^2+3}{3} \right | + \left | \frac{2x^2+\sqrt{3}x}{3} \right | \right \}
    \label{eq2t1}
  \end{equation}
  We can split the inside of equation \ref{eq2t1} as:
  \[ \begin{cases} 
      \frac{8x^2-3}{3}  &-1< x< -\frac{\sqrt{3}}{2} \\
      \frac{-4x^2-2\sqrt{3}x+3}{3}  &-\frac{\sqrt{3}}{2}< x< 0 \\
      \frac{-4x^2+2\sqrt{3}x+3}{3} & 0 < x < \frac{\sqrt{3}}{2} \\
      \frac{8x^2-3}{3} & \frac{\sqrt{3}}{2}<x<1 
    \end{cases}
  \]
  Taking the derivative and making it equal to zero yields
  $x = \pm \sqrt{3}/4, x = 0$. We also consider the problematic points
  $x = \pm 1$ and $x = \pm \sqrt{3}/2$. By substituting these values
  at our function we get that the maximum is attained at $x=\pm 1$
  with the value 5/3.
\end{solution}

%%% Local Variables:
%%% TeX-master: "report.tex"
%%% End:


\begin{problem}
Find the best $L_1$ approximation to the function $f(x) = \sin(x) + \cos(x)$ in the interval $[\pi/3, 5\pi/3]$ from the space of polynomials of degree 2. Show that your result is indeed the unique best $L_1$ approximation to $f$.
\end{problem}

\begin{solution}
Blah blah blah

By the characterization theorem we see that our polynomial is indeed the best approximation:
\begin{equation*}
char theorem eq
\end{equation*}
and as we are approximating from the space of polynomials of degree at most two, which is a Haar space, we have by theorem (14.3 in Powell's book) that our best approximation is unique.
\end{solution}



% \documentclass[a4paper, 11pt]{article}

% \usepackage{amsmath}
% \usepackage{amssymb}
% \usepackage{amsthm}
% \usepackage{graphicx}
% \usepackage{subfig}

% \title{Tasks 2,4,6 of homework 3}
% \author{}
% \date{}
% \begin{document}
% \maketitle

\section*{Task 2}
As we know, the convergence of the Bernstein polynomial approximation is very slow, therefore we need really high degree polynomials in order to achieve the desired accuracy. We first tried programing the algorithm as it is with the binomial coefficient in its expanded form, that is:
\begin{equation*}
\frac{n!}{k!(n-k)!}
\end{equation*}
However with this approach we only reached an accuracy of 0.0258 as we can not compute directly factorials of large numbers (crashed at $n = 172$). Then we turned to compute the binomial coefficients with MATLAB's built in \texttt{nchoosek} function, this did a better job, lowering the error to 0.008, but this is still not enough. Finally (need to do something).

\section*{Task 4}
We have polynomials of the form:
\begin{equation*}
a_1x^1+a_3x^3+a_5x^5+\ldots+a_{n-1}x^{n-1}
\end{equation*}
This is not a Haar space as we can construct polynomials that have more roots than the Haar space can handle. Consider n = 5, then we have a space of dimension $d=3$.
\begin{equation*}
a_1x^1+a_3x^3
\end{equation*}
By definition, all elements of a Haar space of dimension $n^*+1$ have at most $n^*$ roots. In our case, $n^*+1 = 3$ thus, all elements of the space have to have at most $n^*=2$ roots. However let us construct a polynomial with 3 roots. We want a polynomial of the form $(x-x_1)(x-x_2)(x-x_3)$. Expanding this expression we get:
\begin{equation*}
x^3-x^2(x_3+x_2+x_1)+x(x_2x_3+x_1x_3+x_1x_2)-x_1x_2x_3
\end{equation*}
As we want the $x^2$ and the independent term to be gone we get a system of equations:
\begin{align*}
x_3+x_2+x_1 &= 0 \\
x_1x_2x_3 &= 0
\end{align*}
Let $x_3 = 0$, then we get $x_1 = -x_2$, for example let $x_1 = 1$ then we have:
\begin{equation*}
p(x) = x(x-1)(x+1) = x^3-x
\end{equation*}
\qed

\section*{Task 6}
The operator is not linear. Let $f(x) = \sin (x), g(x) = \cos (x)$ and consider the space $\mathcal{P}_0 \subset \mathcal{C}[0, \pi/2]$. It is easy to see that the best minimax approximation to both of these functions is $p(x) = 1/2$ as it satisfies the characterization theorem. However if we take $f(x)+g(x)$ we get that the best minimax approximation is $p(x) = \sin(\pi/4)+\cos(\pi/4)>1$. thus the property:
\begin{equation*}
X(f+g) = X(f) + X(g)
\end{equation*}
is not fulfilled.
\begin{figure}[h]
\centering 
\includegraphics[scale = 0.5]{figtask6.png}
\caption{Plot of the functions and their best minimax approximations}
\label{figtask6}
\end{figure}
\qed
\end{document}

%%% Local Variables:
%%% mode: latex
%%% TeX-master: "report"
%%% End:



% \begin{problem}
  Let $\mathcal{A}$ be the 3-dimensional space of functions on
  $[-1,1]$ composed of two straight line segments joined at $x =
  0$. Calculate the element of $\mathcal{A}$ that minimizes
  \begin{equation}
    \label{eq:integral}
\int_{-1}^1 \lvert x^2 - p(x) \rvert \, \text{dx}, \quad p \in \mathcal{A}.
\end{equation}
\end{problem}


\begin{solution}
$\mathcal{A}$ is a Haar space as if a function in $\mathcal{A}$ has
more than 2 zeros it is identically zero. Which can be taken as the
definition of a Haar space.

We then note that the equation \ref{eq:integral} is the 1-norm and the
element in $\mathcal{A}$ we are searching for is the best $L_1$
approximation from $\mathcal{A}$ to $x^2$. Now we use our dear theorem
14.5 again, giving us the zeros of the error function. These are given
by equation~\ref{eq:zeros}.
\begin{equation}
  \label{eq:zeros}
\zeta_k = \cos{\left (\pi \left(- \frac{k}{4} + \frac{3}{4}\right)
  \right )}
  \Leftrightarrow
  \begin{cases}
    \zeta_0 = - \frac{\sqrt{2}}{2} \\
    \zeta_1 = 0 \\
    \zeta_2 = \frac{\sqrt{2}}{2} \\
  \end{cases}
\end{equation}

By demanding that the error function have these three zeros we get a
candidate for a best approximation $p^*$ in
equation~\ref{eq:pstar}.

\begin{equation}
  \label{eq:pstar}
  p^*(x)  = 
  \begin{cases}
    - \frac{\sqrt{2} x}{2} & \text{for}\: x < 0 \\
    \frac{\sqrt{2} x }{2} & \text{for}\: x \geq 0    
  \end{cases}
\end{equation}

\begin{equation}
  \label{eq:task_4:error}
  x^2 - p^*(x)  = 
  \begin{cases}
    x^{2} + \frac{\sqrt{2} x}{2} & \text{for}\: x < 0 \\
    x^{2} - \frac{\sqrt{2} x}{2} & \text{for}\: x \geq 0
  \end{cases}
\end{equation}

The only thing remaining is to check that the criteria of there being
exactly 3 zeros in the error function is meet. In this case solving
for all the zeros becomes solving a piece wise defined second order
polynomial which is indeed possible. One need not however solve the
equations explicitly. Instead we note that second order polynomials
can have at most 2 zeros and both parts of the error in
equation~\ref{eq:task_4:error} are second order polynomials with one
root in common ($x = 0$). We thus conclude that the error function has
at most 3 zeros and we know by construction that it has 3
zeros. Therefore we know it has {\bf exactly } three zeros. Thus we
have found an element of $\mathcal{A}$ that minimizes
equation~\ref{eq:integral}. Further as $\mathcal{A}$ is an Haar space
the best approximation is unique. 

    
% \begin{figure}[!ht]
%   \centering
%   \includegraphics[scale = 0.5]{task_4_error.png}
%   \caption{Plot of $f - p^*$. Note that there is exactly 3 zeros!.}
%   \label{fig:task_3:error}
% \end{figure}

\end{solution}


%%% Local Variables:
%%% mode: latex
%%% TeX-master: "report"
%%% End:


% \begin{problem}
Show the necessity of taking the \textbf{open} interval in condition (2) (page 77 of Powell's book) by considering the following case. Let $\mathcal{A}$ be spanned by $\Phi_0 = 1, \Phi_1(x) = \cos(2x), \phi_2(x) = \sin(3x), x \in [-\pi/6,\pi/2]$. Show that $\mathcal{A}$ is a Haar space of dimension 3, and that there is no function $p \in \mathcal{A}$ such that $p(-\pi/6) = 0$ but has no other zeros in the given range.
\end{problem}

\begin{solution}
We will provide an example of a function that has a zero at $x = -\pi/6$ and $x = \pi/2$ as we did not succeed in any of our attempts to prove the statement not in the construction of a consistent logic to invert the statement.

We have functions of the form:
\begin{equation}
p(x) = a + b\cos(2x)+c\sin(3x)
\end{equation}
We require a zero at $x = -\pi/6$ thus obtaining the identity:
\begin{equation}
c = a + b/2
\end{equation}
If we choose $b$ to be zero we end up with $c = a$. This leads to:
\begin{equation*}
p(x) = a(1+sin(3x))
\end{equation*}
Which is zero at $x = \pi/2$, which is inside the interval.
\end{solution}

% \begin{problem}
Write
\end{problem}

\begin{solution}
The operator is not linear. Let $f(x) = \sin (x), g(x) = \cos (x)$ and consider the space $\mathcal{P}_0 \subset \mathcal{C}[0, \pi/2]$. It is easy to see that the best minimax approximation to both of these functions is $p(x) = 1/2$ as it satisfies the characterization theorem. However if we take $f(x)+g(x)$ we get that the best minimax approximation is $p(x) = \sin(\pi/4)+\cos(\pi/4)>1$. thus the property:
\begin{equation*}
X(f+g) = X(f) + X(g)
\end{equation*}
is not fulfilled.
\begin{figure}[h]
\centering 
\includegraphics[scale = 0.5]{figtask6.png}
\caption{Plot of the functions and their best minimax approximations}
\label{figtask6}
\end{figure}

\end{solution}

\end{document}

%%% Local Variables:
%%% mode: latex
%%% TeX-master: t
%%% End:
