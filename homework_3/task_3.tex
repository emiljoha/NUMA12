\begin{problem}
  Show that every $p \in \mathcal{P}_n$ has a unique representation as
  $p = a_0 T_o + a_1 T_1 + \cdots + a_n T_n$. Find this representation
  in the case $p(x) = x^2$.
\end{problem}

\begin{solution}
  \begin{equation}
    \label{eq:cheby} T_n(\cos{\theta}) = \cos{n\theta}
  \end{equation} The Chebyshev polynomials as defined in
  equation~\ref{eq:cheby} are quite evidently linearly independent and
  elements in $\mathcal{P}_n$. The set $\{ T_0, \dots , T_n\}$ is
  therefor a set of $n+1$ linearly independent elements in a $n+1$
  dimensional vector space. Thus the Chebyshev polynomials form a basis
  of $\mathcal{P}_n$. Therefore the existence and uniqueness of the
  coefficient $a_0 \dots a_n$ follow from linear algebra of vector
  spaces.

  To actually find the coefficients is harder, as it often is in
  mathematics. We can however get an expression for the coefficients in
  quite a nice form by consider the scalar product defined in exercise 6
  in homework 2. In that exercise we also showed orthogonality. Both
  results are shown in equation~\ref{eq:homework2}.

  \begin{align}
    \label{eq:homework2}
    & \begin{cases}
      <f, g> = & \int_{-1}^{1} \frac{f(x) g(x)}{sqrt{1 - x^2}} dx \\
      <T_i, T_j> = & \delta_{i,j} ( \pi - \frac{\pi}{2}\delta_{i,o} )
    \end{cases} \\
    \nonumber
    \Rightarrow <p, T_i> =  & a_i ( \pi - \frac{\pi}{2}\delta_{i,o} ) 
                              \Leftrightarrow a_i =
                              \frac{<p,T_i>}{\frac{\pi}{2}\delta_{i,o}} \\
    \nonumber 
    <x^n, T_i>  = & \int_{-1}^{1} \frac{x^n T_i(x)}{sqrt{1 - x^2}} dx
    \\
    \label{eq:intexp} 
    a_i = & \frac{\int_0^\pi \cos^{n}\theta \sin{i\theta} d\theta}{\pi
            - \frac{\pi}{2}\delta_{i,o}} \\
  \end{align}

  Equation~\ref{eq:intexp} gives an expression for the coefficients.  
  
\end{solution}




%%% Local Variables:
%%% mode: latex
%%% TeX-master: "report"
%%% End:
