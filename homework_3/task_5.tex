\begin{problem}
Show the necessity of taking the \textbf{open} interval in condition (2) (page 77 of Powell's book) by considering the following case. Let $\mathcal{A}$ be spanned by $\Phi_0 = 1, \Phi_1(x) = \cos(2x), \phi_2(x) = \sin(3x), x \in [-\pi/6,\pi/2]$. Show that $\mathcal{A}$ is a Haar space of dimension 3, and that there is no function $p \in \mathcal{A}$ such that $p(-\pi/6) = 0$ but has no other zeros in the given range.
\end{problem}

\begin{solution}
We will provide an example of a function that has a zero at $x = -\pi/6$ and $x = \pi/2$ as we did not succeed in any of our attempts to prove the statement not in the construction of a consistent logic to invert the statement.

We have functions of the form:
\begin{equation}
p(x) = a + b\cos(2x)+c\sin(3x)
\end{equation}
We require a zero at $x = -\pi/6$ thus obtaining the identity:
\begin{equation}
c = a + b/2
\end{equation}
If we choose $b$ to be zero we end up with $c = a$. This leads to:
\begin{equation*}
p(x) = a(1+sin(3x))
\end{equation*}
Which is zero at $x = \pi/2$, which is inside the interval.
\end{solution}