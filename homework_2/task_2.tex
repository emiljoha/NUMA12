\begin{problem}
Calculate the Lebesgue constants for the Lagrange interpolation
operator with $3, 5, 7, \dots , 21$ equally spaced points, and with the
same number of Chebyshev points.
\end{problem}

%--------------------------------------------------------------------%

\begin{solution}
\begin{tabular}{ l | c | r }
number of points & Equal & Chebyshev \\ \hline
3 & 1.2499997495 & 1.66666666667 \\
5 & 2.20782401056 & 1.988854382 \\
7 & 4.54933607577 & 2.20221455521 \\
9 & 10.9451269881 & 2.36185678777 \\
11 & 29.897758816 & 2.48943037688 \\
12 & 51.2142153912 & 2.54476611865 \\
13 & 89.3241847196 & 2.59567818079 \\
15 & 283.176227752 & 2.68671488157 \\
17 & 934.248871729 & 2.76635333589 \\
19 & 3170.42387769 & 2.83713169974 \\
21 & 10978.7950447 & 2.90082490445 \\
\end{tabular}

Although these values are very reasonable we have reason to trust
their accuracy. When testing the script for the Chebyshev points tabulated in
table 4.5 in the course-book there is a clear deviation. Our values are larger. This is
due to the $\sum_{i=0}^n \|l_i(x)\|$ function having its maxima at the
end points. Inside the interval there were local maxima that where
quite close but just under the tabulated value. This is suggesting
that the Chebyshev points are miss-placed as the errors are not
 ``evened out``. The points are calculated according to equation
(4.27) in the course book and have been compared with the other
calculations to be very precise. 
\begin{equation*}
  x_i = cos\left(\frac{[2(n - 1) +1]\pi}{2(n + 1)}\right)
\end{equation*}
\end{solution}

%%% Local Variables:
%%% TeX-master: "report.tex"
%%% End:
