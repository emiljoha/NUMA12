\begin{problem}
Show that the Chebyshev polynomials are mutually orthogonal relative to the
weight $w(x) = (1-x^2)^{-��1/2}$ on $[-1, 1]$, that is
\begin{equation*}
  \int_{-1}^1 \frac{T_n(x) T_m(x)}{\sqrt{ 1- x^2}} dx = 0
\end{equation*}
\end{problem}

%--------------------------------------------------------------------%

\begin{solution}  
Assume that $n \neq m$. Let us use the change of variables $x = \cos(\theta)$ and the definition of the Chebyshev polynomials
\begin{equation}
\int_{-1}^1 \frac{T_n(x) T_m(x)}{\sqrt{ 1- x^2}} dx = \int_{\pi+2k\pi}^{2k\pi} \cos(n\theta)\cos(m\theta)\frac{1}{\sqrt{1-\cos^2(\theta)}}(-\sin(\theta))d\theta
\label{eq1}
\end{equation}
Now let us use the properties that $\cos^2(\theta)+\sin^2(\theta) = 1$ and $\cos(u)\cos(v) = \frac{1}{2}(\cos(u-v)+\cos(u+v))$ and continue from \ref{eq1}.
\begin{equation*}
\frac{1}{2} \int_{2k\pi}^{\pi+2k\pi} (\cos((n-m)\theta) + \cos((n+m)\theta)) d\theta = \frac{1}{2} \left [ \frac{1}{n-m} \sin ((n-m)\theta) + \frac{1}{n+m} \sin((n+m)\theta) \right ]_{2k\pi}^{\pi + 2k\pi}
\end{equation*}
Which is zero for all $k = 0,1,2,\ldots$

Actually, if $n=m\neq 0$ we have:
\begin{equation*}
\int_{2k\pi}^{\pi+2k\pi} cos^2(\theta) d\theta = \pi/2
\end{equation*}
and if $n=m=0$:
\begin{equation*}
\int_{2k\pi}^{\pi+2k\pi} 1 d\theta = \pi.
\end{equation*}
\end{solution}

%%% Local Variables:
%%% TeX-master: "report.tex"
%%% End:
