\begin{problem}
Let p be the cubic polynomial that interpolates the function values
$f(0)$, $f(1)$, $f(2)$, and $f(3)$. Express $p(6)$ in terms of $f(0)$,
$f(1)$, $f(2)$, $f(3)$, and verify that your formula is correct when
$f$ is the function $ \{ f (x) = (x-3)^3 ; 0 \leq x \leq 6 \}$ What is the
uncertainty in the value of $p(6)$, if the uncertainty in each function
value is $\pm \epsilon$?
\end{problem}

%--------------------------------------------------------------------%

\begin{solution}  
We construct our polynomial using the Lagrange interpolation formula:
\begin{equation*}
p(x) = \sum_{i = 1}^n l_i(x)f(x_i)
\end{equation*}
obtaining the polynomial:
\begin{equation}
p(x) = \frac{(x-1)(x-2)(x-3)}{-6} f(0) + \frac{x(x-2)(x-3)}{2} f(1) + \frac{x(x-1)(x-3)}{-2} f(2) + \frac{x(x-1)(x-2)}{6} f(3)
\label{eq1t5}
\end{equation}
Then we get the formula for $p(6)$:
\begin{equation}
p(6) = -10f(0) + 36f(1) - 45 f(2) + 20 f(3)
\label{eq2t5}
\end{equation}
Now if we use the function $f(x) = (x-3)^3$ we get that $f(0) = -27, f(1) = -8, f(2) = -1, f(3) = 0$. If we substitute these values in equation \ref{eq1t5} we get
\begin{equation*}
p(x) = -\frac{(x-1)(x-2)(x-3)}{-6} 27 - \frac{x(x-2)(x-3)}{2} 8 - \frac{x(x-1)(x-3)}{2}
\end{equation*}
which at $x = 6$ is 27, same as what the formula in equation \ref{eq2t5} gives when we substitute the functional values, therefore our formula is correct for that function.

If we have perturbations in the initial data we would like to know how much our result will be affected by these perturbations:
\begin{equation*}
p(6) = -10(f(0)\pm \epsilon) + 36 (f(1) \pm \epsilon) - 45 (f(2) \pm \epsilon) + 20 (f(3) \pm \epsilon)
\end{equation*}
In the worst case scenario we would have $\epsilon(10+36+45+20)$, that
is, we won't get a worse result than $p(6) + 111\epsilon$. Another
approach would be to take the error as
$\epsilon \sqrt{10^2+36^2+45^2+20^2}$ in any case, we see that if we
get a perturbation that is too large. For example $\epsilon = 0.01$ could
give an error of up to 1.11 units. From this we conclude that the
inherent uncertainty of computation $10^{-16}$ poses no problem but
as the increase in the error is almost two orders of magnitude it can
cause problems if not addressed.  
\end{solution}

%%% Local Variables:
%%% TeX-master: "report.tex"
%%% End:
