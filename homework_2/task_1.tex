\begin{problem}
  Let $-1 \leq x_0 < x_1 < x_2 \leq 1$ and $X$ denote the Lagrange
  interpolator at these points. Show that the minimum value of
  $\lVert X \rVert_\infty$ is 5/4 and is attained when $-x_0 = x_2 \geq 2
  \sqrt{2}/3$ and $x_1 = 0$. Show that if $x_0 , x_1 , x_2$ are
  the zeros of $T_3$, then $\lVert X \rVert_\infty = 5/3$.
\end{problem}

% --------------------------------------------------------------------%

\begin{solution}  
  As we have 3 interpolation points, we know that our polynomial is of
  degree at most 2. The Lebesgue constant of the interpolation
  operator can be written as:
  \begin{equation}
    \lVert X \rVert_{\infty} = \max_{x\in [-1,1]} \sum_{i=0}^2 \lvert l_i(x) \rvert
    \label{defX}
  \end{equation}
  Where $l_i$ are the Lagrange basis polynomials. The task of
  minimizing the max norm of a quantity is the task of balancing
  local maxima so that they are equal. By numerically studying the
  local maxima´s of equation~\ref{defX} we see that the as the distance between two
  points increase, so does the value of the local maxima in
  between. From this realization we see that $x_1= 0$ and
  $-x_0=x_2$ as this is the only setup that preserves the symmetry of
  equation~\ref{defX}. Any other choice would result in one side
  having a higher maxima that the other.

  Now there is only one degree of freedom left. From
  figure~\ref{fig:4edge} in task 4 we see that there is one more
  possible maxima that we have not yet considered. The end
  points. Figure~\ref{fig:4edge} suggests that the end points decrease
  as the end points go further out and become smaller than the
  internal maxima´s around 0.95. A more careful numerical calculation
  gives the results in table~\ref{tab:1}. In table~\ref{tab:1} we see
  something quite spectacular. Between $0.941$ and $0.943$ the maximum
  stop changing and remain stable to an error of $10^{-6}$ for points
  closer to the edges. This value for when the maxima become stable
  perfectly corresponds to $2\frac{\sqrt{2}}{3}$.

  \begin{table}[!ht]
    \caption{Table of global maxima of $\sum_{k=0}^n|l_k(x)|$ with
      three points. $\{-x, 0, x\}$ for $x \in [0.92, 0.96]$ with
      steps length $0.002$}
    \label{tab:1}
    \begin{center}
      \begin{tabular}{ l  c  r }
        $x$ & Maxima \\ \hline
        0.923232323232 & 1.34643211124 \\
        0.925252525253 & 1.33619686886 \\
        0.927272727273 & 1.3260284506 \\
        0.929292929293 & 1.31592627599 \\
        0.931313131313 & 1.30588977089 \\
        0.933333333333 & 1.29591836735 \\
        0.935353535354 & 1.28601150353 \\
        0.937373737374 & 1.27616862366 \\
        0.939393939394 & 1.26638917794 \\
        0.941414141414 & 1.25667262245 \\
        0.943434343434 & 1.24999993218 \\
        0.945454545455 & 1.24999937708 \\
        0.947474747475 & 1.24999992242 \\
        0.949494949495 & 1.24999941213 \\
        0.951515151515 & 1.2499999121 \\
        0.953535353535 & 1.24999944589 \\
        0.955555555556 & 1.24999990124 \\
        0.957575757576 & 1.24999947837 \\
        0.959595959596 & 1.24999988984 \\
        0.961616161616 & 1.24999950961 \\
        0.963636363636 & 1.24999987792 \\
        0.965656565657 & 1.24999953963 \\
      \end{tabular}
    \end{center}
  \end{table}

  If, as stated, we take the points $x_0 = -1, x_1 = 0, x_2 = 1$ and
  expand the LHS of equation \ref{defX} we obtain:
  \begin{equation}
    \max_{x \in [-1,1]} \left \{ \left | \frac{x(x-1)}{2} \right | + \left | -x^2+1 \right | + \left | \frac{x(x+1)}{2} \right | \right \}
    \label{eq1t1}
  \end{equation}
  We can split the inside of equation \ref{eq1t1} as:
  \[ \begin{cases} 
      -x^2-x+1  &-1< x< 0 \\
      -x^2+x+1  &0< x<1 
    \end{cases}
  \]
  Taking the derivative and making it equal to zero yields
  $x = \pm 1/2$. We also consider the problematic points $x = \pm 1$
  and $x = 0$. By substituting these values at our function we get
  that the maximum is attained at $x=\pm 1/2$ with the value 5/4,
  therefore the lower bound for the operator is attained.

  Finally we have that $T_3 = 4x^3-3x$, the roots of $T_3$ are
  $x=0, x=\pm \sqrt{3}/2$. By choosing
  $x_0 = -\frac{\sqrt{3}}{2}, x_1 = 0, x_2 = \frac{\sqrt{3}}{2}$ and
  expanding the LHS of eqaution \ref{defX} we obtain:
  \begin{equation}
    \max_{x \in [-1,1]} \left \{ \left | \frac{2x^2-\sqrt{3}x}{3} \right | + \left | \frac{-4x^2+3}{3} \right | + \left | \frac{2x^2+\sqrt{3}x}{3} \right | \right \}
    \label{eq2t1}
  \end{equation}
  We can split the inside of equation \ref{eq2t1} as:
  \[ \begin{cases} 
      \frac{8x^2-3}{3}  &-1< x< -\frac{\sqrt{3}}{2} \\
      \frac{-4x^2-2\sqrt{3}x+3}{3}  &-\frac{\sqrt{3}}{2}< x< 0 \\
      \frac{-4x^2+2\sqrt{3}x+3}{3} & 0 < x < \frac{\sqrt{3}}{2} \\
      \frac{8x^2-3}{3} & \frac{\sqrt{3}}{2}<x<1 
    \end{cases}
  \]
  Taking the derivative and making it equal to zero yields
  $x = \pm \sqrt{3}/4, x = 0$. We also consider the problematic points
  $x = \pm 1$ and $x = \pm \sqrt{3}/2$. By substituting these values
  at our function we get that the maximum is attained at $x=\pm 1$
  with the value 5/3.
\end{solution}

%%% Local Variables:
%%% TeX-master: "report.tex"
%%% End:
